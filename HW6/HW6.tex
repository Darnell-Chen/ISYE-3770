% Options for packages loaded elsewhere
\PassOptionsToPackage{unicode}{hyperref}
\PassOptionsToPackage{hyphens}{url}
%
\documentclass[
]{article}
\usepackage{amsmath,amssymb}
\usepackage{iftex}
\ifPDFTeX
  \usepackage[T1]{fontenc}
  \usepackage[utf8]{inputenc}
  \usepackage{textcomp} % provide euro and other symbols
\else % if luatex or xetex
  \usepackage{unicode-math} % this also loads fontspec
  \defaultfontfeatures{Scale=MatchLowercase}
  \defaultfontfeatures[\rmfamily]{Ligatures=TeX,Scale=1}
\fi
\usepackage{lmodern}
\ifPDFTeX\else
  % xetex/luatex font selection
\fi
% Use upquote if available, for straight quotes in verbatim environments
\IfFileExists{upquote.sty}{\usepackage{upquote}}{}
\IfFileExists{microtype.sty}{% use microtype if available
  \usepackage[]{microtype}
  \UseMicrotypeSet[protrusion]{basicmath} % disable protrusion for tt fonts
}{}
\makeatletter
\@ifundefined{KOMAClassName}{% if non-KOMA class
  \IfFileExists{parskip.sty}{%
    \usepackage{parskip}
  }{% else
    \setlength{\parindent}{0pt}
    \setlength{\parskip}{6pt plus 2pt minus 1pt}}
}{% if KOMA class
  \KOMAoptions{parskip=half}}
\makeatother
\usepackage{xcolor}
\usepackage[margin=1in]{geometry}
\usepackage{color}
\usepackage{fancyvrb}
\newcommand{\VerbBar}{|}
\newcommand{\VERB}{\Verb[commandchars=\\\{\}]}
\DefineVerbatimEnvironment{Highlighting}{Verbatim}{commandchars=\\\{\}}
% Add ',fontsize=\small' for more characters per line
\usepackage{framed}
\definecolor{shadecolor}{RGB}{248,248,248}
\newenvironment{Shaded}{\begin{snugshade}}{\end{snugshade}}
\newcommand{\AlertTok}[1]{\textcolor[rgb]{0.94,0.16,0.16}{#1}}
\newcommand{\AnnotationTok}[1]{\textcolor[rgb]{0.56,0.35,0.01}{\textbf{\textit{#1}}}}
\newcommand{\AttributeTok}[1]{\textcolor[rgb]{0.13,0.29,0.53}{#1}}
\newcommand{\BaseNTok}[1]{\textcolor[rgb]{0.00,0.00,0.81}{#1}}
\newcommand{\BuiltInTok}[1]{#1}
\newcommand{\CharTok}[1]{\textcolor[rgb]{0.31,0.60,0.02}{#1}}
\newcommand{\CommentTok}[1]{\textcolor[rgb]{0.56,0.35,0.01}{\textit{#1}}}
\newcommand{\CommentVarTok}[1]{\textcolor[rgb]{0.56,0.35,0.01}{\textbf{\textit{#1}}}}
\newcommand{\ConstantTok}[1]{\textcolor[rgb]{0.56,0.35,0.01}{#1}}
\newcommand{\ControlFlowTok}[1]{\textcolor[rgb]{0.13,0.29,0.53}{\textbf{#1}}}
\newcommand{\DataTypeTok}[1]{\textcolor[rgb]{0.13,0.29,0.53}{#1}}
\newcommand{\DecValTok}[1]{\textcolor[rgb]{0.00,0.00,0.81}{#1}}
\newcommand{\DocumentationTok}[1]{\textcolor[rgb]{0.56,0.35,0.01}{\textbf{\textit{#1}}}}
\newcommand{\ErrorTok}[1]{\textcolor[rgb]{0.64,0.00,0.00}{\textbf{#1}}}
\newcommand{\ExtensionTok}[1]{#1}
\newcommand{\FloatTok}[1]{\textcolor[rgb]{0.00,0.00,0.81}{#1}}
\newcommand{\FunctionTok}[1]{\textcolor[rgb]{0.13,0.29,0.53}{\textbf{#1}}}
\newcommand{\ImportTok}[1]{#1}
\newcommand{\InformationTok}[1]{\textcolor[rgb]{0.56,0.35,0.01}{\textbf{\textit{#1}}}}
\newcommand{\KeywordTok}[1]{\textcolor[rgb]{0.13,0.29,0.53}{\textbf{#1}}}
\newcommand{\NormalTok}[1]{#1}
\newcommand{\OperatorTok}[1]{\textcolor[rgb]{0.81,0.36,0.00}{\textbf{#1}}}
\newcommand{\OtherTok}[1]{\textcolor[rgb]{0.56,0.35,0.01}{#1}}
\newcommand{\PreprocessorTok}[1]{\textcolor[rgb]{0.56,0.35,0.01}{\textit{#1}}}
\newcommand{\RegionMarkerTok}[1]{#1}
\newcommand{\SpecialCharTok}[1]{\textcolor[rgb]{0.81,0.36,0.00}{\textbf{#1}}}
\newcommand{\SpecialStringTok}[1]{\textcolor[rgb]{0.31,0.60,0.02}{#1}}
\newcommand{\StringTok}[1]{\textcolor[rgb]{0.31,0.60,0.02}{#1}}
\newcommand{\VariableTok}[1]{\textcolor[rgb]{0.00,0.00,0.00}{#1}}
\newcommand{\VerbatimStringTok}[1]{\textcolor[rgb]{0.31,0.60,0.02}{#1}}
\newcommand{\WarningTok}[1]{\textcolor[rgb]{0.56,0.35,0.01}{\textbf{\textit{#1}}}}
\usepackage{graphicx}
\makeatletter
\def\maxwidth{\ifdim\Gin@nat@width>\linewidth\linewidth\else\Gin@nat@width\fi}
\def\maxheight{\ifdim\Gin@nat@height>\textheight\textheight\else\Gin@nat@height\fi}
\makeatother
% Scale images if necessary, so that they will not overflow the page
% margins by default, and it is still possible to overwrite the defaults
% using explicit options in \includegraphics[width, height, ...]{}
\setkeys{Gin}{width=\maxwidth,height=\maxheight,keepaspectratio}
% Set default figure placement to htbp
\makeatletter
\def\fps@figure{htbp}
\makeatother
\setlength{\emergencystretch}{3em} % prevent overfull lines
\providecommand{\tightlist}{%
  \setlength{\itemsep}{0pt}\setlength{\parskip}{0pt}}
\setcounter{secnumdepth}{-\maxdimen} % remove section numbering
\usepackage{booktabs}
\usepackage{longtable}
\usepackage{array}
\usepackage{multirow}
\usepackage{wrapfig}
\usepackage{float}
\usepackage{colortbl}
\usepackage{pdflscape}
\usepackage{tabu}
\usepackage{threeparttable}
\usepackage{threeparttablex}
\usepackage[normalem]{ulem}
\usepackage{makecell}
\usepackage{xcolor}
\ifLuaTeX
  \usepackage{selnolig}  % disable illegal ligatures
\fi
\usepackage{bookmark}
\IfFileExists{xurl.sty}{\usepackage{xurl}}{} % add URL line breaks if available
\urlstyle{same}
\hypersetup{
  pdftitle={3770 HW6},
  pdfauthor={Darnell Chen},
  hidelinks,
  pdfcreator={LaTeX via pandoc}}

\title{3770 HW6}
\author{Darnell Chen}
\date{2024-10-23}

\begin{document}
\maketitle

\subsection{Problem 6.1.6}\label{problem-6.1.6}

The sample mean of our data set can be calculated by adding all our
values and dividing by the sample size:\textbackslash{}

\(\bar{x} = \sum_{i=1}^n x_i \approx 14.35895\)

On the other hand, our standard deviance is simply the squareroot of our
variance \(s^2\):

\(s^2 = \frac{\sum_{i=1}^n (x_i - \mu)^2}{n-1} \approx 356.4716\)\\
\(s = \sqrt{s^2} = \sqrt{356.4716} \approx 18.88045\)

Thus, we get a mean of approx. 14.35895 and a standard deviation of
approx. 18.88045.

\begin{Shaded}
\begin{Highlighting}[]
\NormalTok{times }\OtherTok{=} \FunctionTok{read.table}\NormalTok{(}\StringTok{"6{-}8.txt"}\NormalTok{, }\AttributeTok{header =} \ConstantTok{TRUE}\NormalTok{)}
\NormalTok{times }\OtherTok{=}\NormalTok{ times}\SpecialCharTok{$}\NormalTok{time}

\FunctionTok{mean}\NormalTok{(times)}
\end{Highlighting}
\end{Shaded}

\begin{verbatim}
## [1] 14.35895
\end{verbatim}

\begin{Shaded}
\begin{Highlighting}[]
\FunctionTok{var}\NormalTok{(times)}
\end{Highlighting}
\end{Shaded}

\begin{verbatim}
## [1] 356.4716
\end{verbatim}

\begin{Shaded}
\begin{Highlighting}[]
\FunctionTok{sqrt}\NormalTok{(}\FunctionTok{var}\NormalTok{(times))}
\end{Highlighting}
\end{Shaded}

\begin{verbatim}
## [1] 18.88045
\end{verbatim}

\(~\)

\pagebreak

\subsection{Problem 6.2.4}\label{problem-6.2.4}

The two middle numbers in our data set are both 90.4. Since
\(\frac{90.4 + 90.4}{2} = 90.4\), our median is thus, 90.4.

\begin{Shaded}
\begin{Highlighting}[]
\NormalTok{rating }\OtherTok{=} \FunctionTok{read.table}\NormalTok{(}\StringTok{"6{-}30.txt"}\NormalTok{, }\AttributeTok{header =} \ConstantTok{TRUE}\NormalTok{)}
\NormalTok{rating }\OtherTok{=}\NormalTok{ rating}\SpecialCharTok{$}\NormalTok{Rating}

\FunctionTok{median}\NormalTok{(rating)}
\end{Highlighting}
\end{Shaded}

\begin{verbatim}
## [1] 90.4
\end{verbatim}

Our Quartiles:

\begin{Shaded}
\begin{Highlighting}[]
\NormalTok{quantiles }\OtherTok{=} \FunctionTok{quantile}\NormalTok{(rating,}\AttributeTok{type=}\DecValTok{6}\NormalTok{)}
\NormalTok{quantiles}
\end{Highlighting}
\end{Shaded}

\begin{verbatim}
##      0%     25%     50%     75%    100% 
##  83.400  88.575  90.400  92.200 100.300
\end{verbatim}

Our Stem and Leaf Plot:

\begin{Shaded}
\begin{Highlighting}[]
\FunctionTok{stem}\NormalTok{(rating, }\AttributeTok{scale =} \DecValTok{2}\NormalTok{)}
\end{Highlighting}
\end{Shaded}

\begin{verbatim}
## 
##   The decimal point is at the |
## 
##    83 | 4
##    84 | 33
##    85 | 3
##    86 | 777
##    87 | 456789
##    88 | 23334556679
##    89 | 0233678899
##    90 | 0111344456789
##    91 | 0001112256688
##    92 | 22236777
##    93 | 023347
##    94 | 2247
##    95 | 
##    96 | 15
##    97 | 
##    98 | 8
##    99 | 
##   100 | 3
\end{verbatim}

\(~\)

\pagebreak

\subsection{Problem 6.3.2}\label{problem-6.3.2}

Our Histogram:

\begin{Shaded}
\begin{Highlighting}[]
\NormalTok{bin}\OtherTok{=}\FunctionTok{seq}\NormalTok{(}\FunctionTok{min}\NormalTok{(rating),}\FunctionTok{max}\NormalTok{(rating),}\AttributeTok{by=}\NormalTok{(}\FunctionTok{max}\NormalTok{(rating)}\SpecialCharTok{{-}}\FunctionTok{min}\NormalTok{(rating))}\SpecialCharTok{/}\DecValTok{8}\NormalTok{)}
\NormalTok{freqs }\OtherTok{=} \FunctionTok{hist}\NormalTok{(rating, }\AttributeTok{breaks=}\NormalTok{bin, }\AttributeTok{label=}\ConstantTok{TRUE}\NormalTok{, }\AttributeTok{right=}\ConstantTok{FALSE}\NormalTok{, }\AttributeTok{col=}\FunctionTok{rainbow}\NormalTok{(}\DecValTok{8}\NormalTok{), }\AttributeTok{ylim=}\FunctionTok{c}\NormalTok{(}\DecValTok{0}\NormalTok{, }\DecValTok{32}\NormalTok{))}
\end{Highlighting}
\end{Shaded}

\includegraphics{HW6_files/figure-latex/unnamed-chunk-5-1.pdf}

\begin{Shaded}
\begin{Highlighting}[]
\NormalTok{freqs}
\end{Highlighting}
\end{Shaded}

\begin{verbatim}
## $breaks
## [1]  83.4000  85.5125  87.6250  89.7375  91.8500  93.9625  96.0750  98.1875
## [9] 100.3000
## 
## $counts
## [1]  4  6 20 30 14  4  2  2
## 
## $density
## [1] 0.02309136 0.03463703 0.11545678 0.17318516 0.08081974 0.02309136 0.01154568
## [8] 0.01154568
## 
## $mids
## [1] 84.45625 86.56875 88.68125 90.79375 92.90625 95.01875 97.13125 99.24375
## 
## $xname
## [1] "rating"
## 
## $equidist
## [1] TRUE
## 
## attr(,"class")
## [1] "histogram"
\end{verbatim}

Our Frequency Table:

\begin{Shaded}
\begin{Highlighting}[]
\NormalTok{str}\OtherTok{=}\ConstantTok{NULL}
\ControlFlowTok{for}\NormalTok{ (i }\ControlFlowTok{in} \DecValTok{1}\SpecialCharTok{:}\DecValTok{7}\NormalTok{) \{}
\NormalTok{str }\OtherTok{=} \FunctionTok{c}\NormalTok{(str, }\FunctionTok{paste}\NormalTok{(freqs}\SpecialCharTok{$}\NormalTok{breaks[i], }\StringTok{"$}\SpecialCharTok{\textbackslash{}\textbackslash{}}\StringTok{le x \textless{} $"}\NormalTok{, freqs}\SpecialCharTok{$}\NormalTok{breaks[i}\SpecialCharTok{+}\DecValTok{1}\NormalTok{]))}
\NormalTok{\}}
\CommentTok{\# Handle the last interval with inclusive bounds}
\NormalTok{str }\OtherTok{=} \FunctionTok{c}\NormalTok{(str, }\FunctionTok{paste}\NormalTok{(freqs}\SpecialCharTok{$}\NormalTok{breaks[}\DecValTok{8}\NormalTok{], }\StringTok{"$}\SpecialCharTok{\textbackslash{}\textbackslash{}}\StringTok{le x }\SpecialCharTok{\textbackslash{}\textbackslash{}}\StringTok{le $"}\NormalTok{, freqs}\SpecialCharTok{$}\NormalTok{breaks[}\DecValTok{9}\NormalTok{]))}
\NormalTok{df }\OtherTok{=} \FunctionTok{data.frame}\NormalTok{(}\AttributeTok{Class=}\NormalTok{str, }\AttributeTok{Index=}\NormalTok{freqs}\SpecialCharTok{$}\NormalTok{counts)}
\FunctionTok{library}\NormalTok{(knitr)}
\FunctionTok{library}\NormalTok{(kableExtra)}
\FunctionTok{kable}\NormalTok{(df, }\StringTok{"latex"}\NormalTok{, }\AttributeTok{align=}\StringTok{"c"}\NormalTok{, }\AttributeTok{escape =}\NormalTok{ F, }\AttributeTok{caption=}\StringTok{"Hint"}\NormalTok{) }\SpecialCharTok{\%\textgreater{}\%}
  \FunctionTok{kable\_styling}\NormalTok{(}\AttributeTok{latex\_options =} \StringTok{"hold\_position"}\NormalTok{)}
\end{Highlighting}
\end{Shaded}

\begin{table}[!h]
\centering
\caption{\label{tab:unnamed-chunk-6}Hint}
\centering
\begin{tabular}[t]{c|c}
\hline
Class & Index\\
\hline
83.4 $\le x < $ 85.5125 & 4\\
\hline
85.5125 $\le x < $ 87.625 & 6\\
\hline
87.625 $\le x < $ 89.7375 & 20\\
\hline
89.7375 $\le x < $ 91.85 & 30\\
\hline
91.85 $\le x < $ 93.9625 & 14\\
\hline
93.9625 $\le x < $ 96.075 & 4\\
\hline
96.075 $\le x < $ 98.1875 & 2\\
\hline
98.1875 $\le x \le $ 100.3 & 2\\
\hline
\end{tabular}
\end{table}

\(~\)

\pagebreak

\subsection{Problem 6.4.9}\label{problem-6.4.9}

Based on my boxplot below, I think it's hard to definitively tell if the
treatment is effective or not in gene expression. Because while it does
have the smallest range - and practically no outliers - there's still a
ton of difference in variance between the control groups. However, the
variance and outliers in the High Dosage group itself definitely
displayed a much smaller range compared to the other groups. So if I had
to choose a definitive answer, I'd say that the high dosage treatment
definitely helped in minimizing gene expression.

\begin{Shaded}
\begin{Highlighting}[]
\NormalTok{treatmentData }\OtherTok{=} \FunctionTok{read.table}\NormalTok{(}\StringTok{"6{-}81.txt"}\NormalTok{, }\AttributeTok{header=}\ConstantTok{TRUE}\NormalTok{)}
\FunctionTok{attach}\NormalTok{(treatmentData)}
\FunctionTok{boxplot}\NormalTok{(Expression}\SpecialCharTok{\textasciitilde{}}\NormalTok{Group, }\AttributeTok{col=}\FunctionTok{rainbow}\NormalTok{(}\DecValTok{10}\NormalTok{))}
\end{Highlighting}
\end{Shaded}

\includegraphics{HW6_files/figure-latex/unnamed-chunk-7-1.pdf} \(~\)

\pagebreak

\subsection{Problem 6.7.2}\label{problem-6.7.2}

Based on the normal distribution line below, it would be reasonable to
assume that the octane rating follows a normal distribution. This is
because most of our ratings/points follow the expected normal
distribution line.

\begin{Shaded}
\begin{Highlighting}[]
\FunctionTok{qqnorm}\NormalTok{(rating)}
\FunctionTok{qqline}\NormalTok{(rating, }\AttributeTok{col=}\StringTok{"red"}\NormalTok{)}
\end{Highlighting}
\end{Shaded}

\includegraphics{HW6_files/figure-latex/unnamed-chunk-8-1.pdf}

\end{document}
